\PassOptionsToPackage{unicode=true}{hyperref} % options for packages loaded elsewhere
\PassOptionsToPackage{hyphens}{url}
%
\documentclass[]{article}
\usepackage{lmodern}
\usepackage{amssymb,amsmath}
\usepackage{ifxetex,ifluatex}
\usepackage{fixltx2e} % provides \textsubscript
\ifnum 0\ifxetex 1\fi\ifluatex 1\fi=0 % if pdftex
  \usepackage[T1]{fontenc}
  \usepackage[utf8]{inputenc}
  \usepackage{textcomp} % provides euro and other symbols
\else % if luatex or xelatex
  \usepackage{unicode-math}
  \defaultfontfeatures{Ligatures=TeX,Scale=MatchLowercase}
\fi
% use upquote if available, for straight quotes in verbatim environments
\IfFileExists{upquote.sty}{\usepackage{upquote}}{}
% use microtype if available
\IfFileExists{microtype.sty}{%
\usepackage[]{microtype}
\UseMicrotypeSet[protrusion]{basicmath} % disable protrusion for tt fonts
}{}
\IfFileExists{parskip.sty}{%
\usepackage{parskip}
}{% else
\setlength{\parindent}{0pt}
\setlength{\parskip}{6pt plus 2pt minus 1pt}
}
\usepackage{hyperref}
\hypersetup{
            pdftitle={Power of Wald test for interaction},
            pdfauthor={Ziang Zhang},
            pdfborder={0 0 0},
            breaklinks=true}
\urlstyle{same}  % don't use monospace font for urls
\usepackage[margin=1in]{geometry}
\usepackage{graphicx,grffile}
\makeatletter
\def\maxwidth{\ifdim\Gin@nat@width>\linewidth\linewidth\else\Gin@nat@width\fi}
\def\maxheight{\ifdim\Gin@nat@height>\textheight\textheight\else\Gin@nat@height\fi}
\makeatother
% Scale images if necessary, so that they will not overflow the page
% margins by default, and it is still possible to overwrite the defaults
% using explicit options in \includegraphics[width, height, ...]{}
\setkeys{Gin}{width=\maxwidth,height=\maxheight,keepaspectratio}
\setlength{\emergencystretch}{3em}  % prevent overfull lines
\providecommand{\tightlist}{%
  \setlength{\itemsep}{0pt}\setlength{\parskip}{0pt}}
\setcounter{secnumdepth}{5}
% Redefines (sub)paragraphs to behave more like sections
\ifx\paragraph\undefined\else
\let\oldparagraph\paragraph
\renewcommand{\paragraph}[1]{\oldparagraph{#1}\mbox{}}
\fi
\ifx\subparagraph\undefined\else
\let\oldsubparagraph\subparagraph
\renewcommand{\subparagraph}[1]{\oldsubparagraph{#1}\mbox{}}
\fi

% set default figure placement to htbp
\makeatletter
\def\fps@figure{htbp}
\makeatother


\title{Power of Wald test for interaction}
\author{Ziang Zhang}
\date{09/03/2021}

\begin{document}
\maketitle

{
\setcounter{tocdepth}{2}
\tableofcontents
}
\hypertarget{power-of-wald-test}{%
\section{Power of Wald test:}\label{power-of-wald-test}}

\hypertarget{for-one-parameter}{%
\subsection{For one parameter:}\label{for-one-parameter}}

If we consider using Wald test to test the null hypothesis
\(H_0:\theta=\theta_0\), the test statistic will be
\[\sqrt{\frac{n}{I^{-1}(\hat{\theta})}}\bigg(\hat{\theta} - \theta_0\bigg)\]
where \(\hat{\theta}\) is the MLE of \(\theta\) and
\(I^{-1}(\hat{\theta})\) is inverse of the fisher information evaluated
at the MLE.

Under null hypothesis, this test statistic follows a standard normal
distribution. If the alternative hypothesis
\(H_a:\theta=\theta_1 \neq \theta_0\), then the power of our test can be
computed as: \[1-\Phi(\Delta+z_{a/2}) + \Phi(\Delta-z_{a/2})\] where
\(\Delta = \sqrt{n I(\theta_1)}(\theta_0-\theta_1)\).

The important information above is that the power of Wald test will
depend on several things at the same time:

\begin{itemize}
\tightlist
\item
  The difference of \(\theta_0 - \theta_1\)
\item
  The sample size \(n\)
\item
  The \texttt{true} information \(I(\theta_1)\)
\end{itemize}

\end{document}
