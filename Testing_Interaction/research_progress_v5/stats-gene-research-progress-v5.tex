% Options for packages loaded elsewhere
\PassOptionsToPackage{unicode}{hyperref}
\PassOptionsToPackage{hyphens}{url}
%
\documentclass[
]{article}
\usepackage{lmodern}
\usepackage{amssymb,amsmath}
\usepackage{ifxetex,ifluatex}
\ifnum 0\ifxetex 1\fi\ifluatex 1\fi=0 % if pdftex
  \usepackage[T1]{fontenc}
  \usepackage[utf8]{inputenc}
  \usepackage{textcomp} % provide euro and other symbols
\else % if luatex or xetex
  \usepackage{unicode-math}
  \defaultfontfeatures{Scale=MatchLowercase}
  \defaultfontfeatures[\rmfamily]{Ligatures=TeX,Scale=1}
\fi
% Use upquote if available, for straight quotes in verbatim environments
\IfFileExists{upquote.sty}{\usepackage{upquote}}{}
\IfFileExists{microtype.sty}{% use microtype if available
  \usepackage[]{microtype}
  \UseMicrotypeSet[protrusion]{basicmath} % disable protrusion for tt fonts
}{}
\makeatletter
\@ifundefined{KOMAClassName}{% if non-KOMA class
  \IfFileExists{parskip.sty}{%
    \usepackage{parskip}
  }{% else
    \setlength{\parindent}{0pt}
    \setlength{\parskip}{6pt plus 2pt minus 1pt}}
}{% if KOMA class
  \KOMAoptions{parskip=half}}
\makeatother
\usepackage{xcolor}
\IfFileExists{xurl.sty}{\usepackage{xurl}}{} % add URL line breaks if available
\IfFileExists{bookmark.sty}{\usepackage{bookmark}}{\usepackage{hyperref}}
\hypersetup{
  pdftitle={Progress report: Detecting interaction with unknown environmental covariate},
  pdfauthor={Ziang Zhang},
  hidelinks,
  pdfcreator={LaTeX via pandoc}}
\urlstyle{same} % disable monospaced font for URLs
\usepackage[margin=1in]{geometry}
\usepackage{graphicx,grffile}
\makeatletter
\def\maxwidth{\ifdim\Gin@nat@width>\linewidth\linewidth\else\Gin@nat@width\fi}
\def\maxheight{\ifdim\Gin@nat@height>\textheight\textheight\else\Gin@nat@height\fi}
\makeatother
% Scale images if necessary, so that they will not overflow the page
% margins by default, and it is still possible to overwrite the defaults
% using explicit options in \includegraphics[width, height, ...]{}
\setkeys{Gin}{width=\maxwidth,height=\maxheight,keepaspectratio}
% Set default figure placement to htbp
\makeatletter
\def\fps@figure{htbp}
\makeatother
\setlength{\emergencystretch}{3em} % prevent overfull lines
\providecommand{\tightlist}{%
  \setlength{\itemsep}{0pt}\setlength{\parskip}{0pt}}
\setcounter{secnumdepth}{5}

\title{Progress report: Detecting interaction with unknown environmental
covariate}
\author{Ziang Zhang}
\date{15/10/2020}

\begin{document}
\maketitle

\hypertarget{the-underlying-model}{%
\section{The Underlying Model:}\label{the-underlying-model}}

For binary response variable, it is often assumed that the response
variable \(y_i\) conditioning on the regressors \(G_i,Z_i\) come from a
latent model such that: \begin{equation}\label{eqn:latentformulation}
\begin{aligned}
Y_i^* &= \beta_0 + \beta_G G_i + \beta_Z Z_i + \epsilon_i \\
Y_i &= I\{Y_i^*>0\} \\
\end{aligned}
\end{equation}

The unobserved latent variable \(Y_i^*\) determines whether the observed
response variable \(Y_i\) is 0 or 1. The error term \(\epsilon_i\) in
\(Y_i^*\) needs to have a completely known distribution, which can be
\(\text{N}(0,1)\) for the model to become a probit model, or a logistic
distribution with mean 0 and variance 3.28 for the model to become a
logistic regression model.

Here the regressor \(G_i\) represents the allele of interest, and the
regressor \(Z_i\) is any regressor that can be non-genetic. For now on,
we will assume the model is probit for simplicity, unless otherwise
indicated.

Similarly, we can have a Genotypic Model defined as:\\
\begin{equation}\label{eqn:latentformulationGeno}
\begin{aligned}
Y_i^* &= \beta_0 + \beta_{G1} I(G_i = 1) + \beta_{G2} I(G_i = 2) + \beta_Z Z_i + \epsilon_i \\
Y_i &= I\{Y_i^*>0\} \\
\end{aligned}
\end{equation}

The Genotypic Model has higher degree of freedom than the additive model
due to the extra regression parameter.

\hypertarget{when-the-true-model-does-contain-gene-environment-interaction}{%
\subsection{When the true model does contain gene-environment
interaction}\label{when-the-true-model-does-contain-gene-environment-interaction}}

Assume for simplicity that \(E_i\) the environmental variable has a
normal distribution with mean \(\mu_E\) and variance \(\sigma_E^2\), and
suppose that the true underlying model is:
\begin{equation}\label{eqn:probitModelWithInteraction}
\begin{aligned}
Y_i^* &= \beta_0 + \beta_G G_i + \beta_Z Z_i + \beta_E + E_i + \beta_{G\times E} G_i \times E_i + \epsilon_i \\
Y_i &= I\{Y_i^*>0\} \\
\epsilon_i &\sim \text{N}(0,1)
\end{aligned}
\end{equation}

Furthermore, we can compute that:
\begin{equation}\label{eqn:probitModelWithInteraction_MeanVar}
\begin{aligned}
\text{E}(Y_i^*|G_i,Z_i) &= \beta_0 + \beta_E \mu_E + (\beta_G + \beta_{G\times E} \mu_E)G_i + \beta_Z Z_i \\
\text{Var}(Y_i^*|G_i,Z_i) &= (\beta_{G\times E} G_i)^2 \sigma_E^2 + \beta_E^2 \sigma_E^2 + 1 \\
Y_i^*|G_i, Z_i &\sim \text{N}\bigg(\beta_0 + \beta_E \mu_E + (\beta_G + \beta_{G\times E} \mu_E)G_i + \beta_Z Z_i, (\beta_{G\times E} G_i)^2 \sigma_E^2 + \beta_E^2 \sigma_E^2 + 1 \bigg)
\end{aligned}
\end{equation}

That implies that the probability we get a case for different levels of
\(G_i\) and \(Z_i\) will be:
\begin{equation}\label{eqn:probitModelWithInteraction_Prob} 
\begin{aligned} 
\text{P}(Y = 1 | G_i, Z_i) &= \text{P}(Y^* > 0| G_i, Z_i) \\ 
                           &= \text{P}(\frac{Y^*  - \text{E}(Y^* |G_i,Z_i)}{\sqrt{\text{Var}(Y^* |G_i,Z_i)}} > \frac{-\text{E}(Y^* |G_i,Z_i)}{\sqrt{\text{Var}(Y^* |G_i,Z_i)}}) \\
                           &= \Phi \bigg( \frac{\text{E}(Y^* |G_i,Z_i)}{\sqrt{\text{Var}(Y^* |G_i,Z_i)}} \bigg)
\end{aligned}
\end{equation}

Therefore, applying the inverse CDF on both sides, we get
\[\Phi^{-1} \bigg(\text{P}(Y = 1 | G, Z) \bigg) = \frac{\beta_0+\beta_E \mu_E+(\beta_G + \beta_{G\times E} \mu_E)G_i + \beta_Z Z}{\sqrt{(\beta_{G\times E}^2 G_i^2 \sigma_E^2 + \beta_E^2 \sigma_E^2 + 1)}} \]

This is only a linear function of \(G_i\) when the interaction parameter
\(\beta_{G\times E} = 0\), and the slope of Z is constant across
different genes only when \(\beta_{G\times E}\) is zero.

\begin{enumerate}
\item If the true underlying model also contains another regressor $W$ but $W$ is uncorrelated with $G$ for example. Then even though ignoring that regressor breaks the structural assumption of probit model, so that the fitted model without $W$ is no longer a probit model (since now $\epsilon$ does not follow standard normal), but $\Phi^{-1}(\text{P}(Y_i = 1|G_i,Z_i))$ will still be a linear function of $G_i$. So detecting based on the linearity of $\Phi^{-1}\text{P}$ will not be affected by omitted exogenous regressors.
\item Since $P(Y_i = 1|G_i,Z_i)$ is actually unknown in practice, we can estimate it using the sample proportion $\hat{P}(Y = 1|G = g,Z = z) = \frac{\sum_{i=1}^{n} \text{I}\{y_i =1,G_{i} = g, Z_{i} = z\}}{\sum_{i=1}^{n}  \text{I}\{G_{i} = g, Z_{i} = z\}}$. We shouldn't use the fitted model to estimate them since our fitted model may be wrong.
\item The reason we used probit model instead of logistic model here is that assuming $E$ follows normal distribution, $Y^*|G,Z$ will still be normal if we omit the interaction term, since linear combination of normal is normal. But assuming $E$ follows logistic distribution does not imply that $Y^*|G,Z$ will be logistically distributed as logistic distribution is not closed under linear combination. However, based on the literature, it seems like probit model and logistic model have really closed results in real applications.
\end{enumerate}

If the model is Genotypic instead: \begin{equation}\label{eqn:genointer}
\begin{aligned}
Y_i^* &= \beta_0 + \beta_{G1} I(G_i = 1) + \beta_{G2} I(G_i = 2) + \beta_Z Z_i + \beta_E E_i + \beta_{G1E} I(G_i = 1) \times E_i + \beta_{G2E} I(G_i = 2) \times E_i  + \epsilon_i \\
\end{aligned}
\end{equation}

then we can derive the following:
\[\Phi^{-1} \bigg(\text{P}(Y = 1 | G, Z) \bigg) = \frac{\beta_0+\beta_E \mu_E+(\beta_{G1} + \beta_{G1E} \mu_E)I(G_i = 1)+(\beta_{G2} + \beta_{G2E} \mu_E)I(G_i = 2) + \beta_Z Z}{\sqrt{(\beta_{G1E}^2 I(G_i = 1) \sigma_E^2 +\beta_{G2E}^2 I(G_i = 2) \sigma_E^2 + \beta_E^2 \sigma_E^2 + 1)}} \]
In this case, the model will still be linear in
\(I(G_i = 1), I(G_i = 2)\) because of the extra parameter, just with
different regression parameters. But the slope of Z will continue to
differ between different Genetic types unless the there are no
interaction effects \(\beta_{G1E}\) and \(\beta_{G2E}\).

\hypertarget{method-for-additive-model}{%
\section{Method for Additive Model:}\label{method-for-additive-model}}

In this section, I will present two methods for the detection of
interaction effect when the true model is additive.

\hypertarget{testing-of-linearity}{%
\subsection{Testing of Linearity:}\label{testing-of-linearity}}

Recall that when the model is additive, then:
\[\Phi^{-1} \bigg(\text{P}(Y = 1 | G, Z) \bigg) = \frac{\beta_0+\beta_E \mu_E+(\beta_G + \beta_{G\times E} \mu_E)G_i + \beta_Z Z}{\sqrt{(\beta_{G\times E}^2 G_i^2 \sigma_E^2 + \beta_E^2 \sigma_E^2 + 1)}} \]

This method relies on the checking of linearity of \(\Phi^{-1}(P)\), so
the test statistics will also be focusing on the detection of linearity.
We first derive the test statistic using the Delta method on sample
proportions, and then we will generalize this test statistic using a
linear regression framework.

\hypertarget{derivation-using-delta-method}{%
\subsubsection{Derivation using Delta
Method:}\label{derivation-using-delta-method}}

If the two competing models are: \begin{equation}
\begin{aligned}
1. Y_i ^{*} = \beta_0 + \beta_G G_i + \beta_E E_i + \epsilon_i \\
2. Y_i ^{*} = \beta_0 + \beta_G G_i + \beta_E E_i + \beta_{G\times E}G_i \times E_i + \epsilon_i \\
\end{aligned}
\end{equation}

Let \(p_i = P(Y = 1|G=i)\), which can be approximated by the sample
proportion \(\hat{p_i}\). We know that
\[\Phi^{-1}(\hat{p_i}) \sim N\bigg(\Phi^{-1}(p_i), \frac{1}{\phi(\Phi^{-1}(p_i))^2} \frac{p_i(1-p_i)}{n_i}\bigg)\]
Here \(n_i\) denotes the number of \(G = i\) in our dataset. Let
\(v_i = \frac{1}{\phi(\Phi^{-1}(\hat{p_i}))^2} \frac{\hat{p_i}(1-\hat{p_i})}{n_i}\)
denote the estimate of \(Var(\Phi^{-1}(\hat{p_i}))\).

Let
\(S = \Phi^{-1}(\hat{p_2}) - 2\Phi^{-1}(\hat{p_1}) + \Phi^{-1}(\hat{p_0})\),
and \(T = \frac{S^2}{v_o+4v_1+v_2}\) be our test statistic. If 1 is the
true model, then \(T \sim X^2_{1}\).

\hypertarget{two-stage-regression-framework}{%
\subsubsection{Two Stage regression
framework:}\label{two-stage-regression-framework}}

The above relationship can be generalized using the idea of a two stage
regression. In the first stage, we consider the following generalized
linear regression:
\[ E(Y_i|G_i) = \Phi\bigg(\beta_0+ \beta_1 \text{I}(G_i =1) + \beta_2 \text{I}(G_i =2)\bigg)\]

From this generalized linear regression, we obtained the fitted linear
predictors
\(\hat{\eta_i} = \hat{\beta_0}+\hat{\beta_1}\text{I}(G_i =1) + \hat{\beta_2} \text{I}(G_i =2)\).
Then, we fit the following weighted least square regression in the
second stage:
\[ \hat{\eta_i} = \gamma_0 +\gamma_1  \text{I}(G_i =1) + \gamma_2 \text{I}(G_i =2) +\epsilon_i\]
where
\(\epsilon_i \sim N\bigg(0,\frac{\Phi(\hat{\eta_i})(1-\Phi(\hat{\eta_i}))}{\phi(\hat{\eta_i})^2} \bigg)\).
Define the weight matrix
\(W = \text{diag}\{\frac{\Phi(\hat{\eta_i})(1-\Phi(\hat{\eta_i}))}{\phi(\hat{\eta_i})^2}\}\),
then our previous test statistic is just the test statistic of Wald test
of \(H_0:\gamma_2 = 2\gamma_1\).

The intuition behind how this approach works to detect the interaction
between \(G\) and \(E\) can be thought as following. In the first
regression, we are actually trying to get an estimate of
\(\eta = \frac{\text{E}(Y^*|G)}{\sqrt{\text{Var}(Y^*|G)}}\). Recall from
the first section, if there is interaction between \(G\) and \(E\), then
\(\eta\) will be non-linear in \(G\) due to the fact that
\(\text{Var}(Y^*|G)\) will be non-constant (We cannot directly test on
whether \(Y^*\) has constant variance because it is not estimable).
Therefore, if we observe that \(\hat{\eta}\) is linear in \(G\), which
means \(\gamma_2 = \gamma_1\), then we can be reasonable confident that
there is no interaction present in the model.

We can use the first stage regression to obtain accurate estimate of
\(\eta\) because regardless whether \(\eta\) is linear in G, we know
\(\eta\) can be written as a linear model with \(\text{I}(G_i =1)\) and
\(\text{I}(G_i =2)\). This suggests, if we have another covariate \(Z\)
in the model, we can just do the two stage regressions with all of
\(\text{I}(G_i =j)\) and \(\text{I}(G_i =j)Z_i\) (Then in our null
hypothesis, besides that \(\gamma_2 = 2 \gamma_1\), we can also add
additional constraints that all of \(\text{I}(G_i =j)Z_i\) should have
the same slopes to increase our power).

\hypertarget{testing-using-random-slopes}{%
\subsection{Testing using random
slopes:}\label{testing-using-random-slopes}}

First, let's rewrite our previous latent variable specification:
\begin{equation}\label{eqn:latentformulationRandomSlope}
\begin{aligned}
Y_i^* &= \beta_0 + \beta_G G_i + \beta_Z Z_i + \beta_{G\times E} G_i \times E_i + \epsilon_i \\
      &= \beta_0 + \beta_G G_i + \beta_Z Z_i + U_i * G_i + \epsilon_i \\
Y_i &= I\{Y_i^*>0\} \\
U_i &= \beta_{G\times E} * E_i
\end{aligned}
\end{equation}

Here \(U_i\) can be thought as a random effect (random slope), being
drawn from distribution \(\text{N}(0,\sigma_u^2)\). Notice that
\(\sigma_u^2 = \beta_{G\times E}^2 \sigma_E^2\). Therefore, testing for
\(\beta_{G\times E} =0\) is equivalent to testing \(\sigma_u^2 = 0\) for
the random effects. In this case, we do not need to restrict our
distribution to the probit model anymore. Since both probit model and
logistic model are flexible enough to incorporate an observations-level
random slopes. (There shouldn't be any identifiability problem with have
too many random slopes(same number as observations), as including an
observations-level random intercepts is a common trick to account for
overdispersion in Poisson regression.)

In this case, we can use the likelihood Ratio test to test the model
with and without the random slopes, with correction to the boundary.
Therefore, the final test statistic will be
\(-2\Lambda \sim 0.5X^2_1 + 0.5X^2_0\) under null hypothesis.

\hypertarget{difference-between-two-potential-methods}{%
\subsubsection{Difference between two potential
methods}\label{difference-between-two-potential-methods}}

\begin{enumerate}
\item The first method relies on the assumption that the true underlying model is probit model, and the distribution of $E$ is normal. These assumptions shouldn't be too restrictive as it is said in the literature that probit model and logistic model tend to give similar results. However, the second method can be used for both probit model and logistic model. The only assumption in the second method is that $E$ follows a normal distribution.
\item The next step for the first method is to develop a test statistic for testing the linearity. While for the second method, it seems like there are plenty of tools of testing at boundary to test $\sigma_u = 0$, using likelihood ratio. It seems like in the second method, jointly testing for the main effect and interaction effect 
\item For the simulations of sample size 300000, the first method is very efficient to compute as it basically just computes nine sample proportions and compute their difference. If we can find a good test statistic for this, the hypothesis testing will be efficient to carry out and scale to larger sample. The second method takes a very long time to converge when the interaction is actually present in the model, and lme4 tends to give some warnings about the potential convergence problems if a probit model is fitted and underlying model has the interaction effect. For a larger sample with more regressors, the computational loads will be bigger for the second method.
\item Note that for the second method to work, we need to assume there is no main effect of the environment variable $E_i$, i.e. $\beta_E = 0$ need to be assumed in the model, because otherwise we will need not only individual level slopes but also individual level intercepts to capture the true model, which will cause identifiability issues.
\end{enumerate}

\hypertarget{method-for-genotypic-model}{%
\section{Method for Genotypic Model:}\label{method-for-genotypic-model}}

\hypertarget{auxiliary-variable-method}{%
\subsection{Auxiliary variable
method:}\label{auxiliary-variable-method}}

Recall for a Genotypic Model with interaction like below:
\begin{equation}\label{eqn:genointer}
\begin{aligned}
Y_i^* &= \beta_0 + \beta_{G1} I(G_i = 1) + \beta_{G2} I(G_i = 2) + \beta_Z Z_i + \beta_E E_i + \beta_{G1E} I(G_i = 1) \times E_i + \beta_{G2E} I(G_i = 2) \times E_i  + \epsilon_i \\
\end{aligned}
\end{equation}

We can derive that: \begin{equation}\label{eqn:RStest}
\begin{aligned}
\Phi^{-1} \bigg(\text{P}(Y = 1 | G, Z) \bigg) &= \frac{\beta_0+\beta_E \mu_E+(\beta_{G1} + \beta_{G1E} \mu_E)I(G = 1)+(\beta_{G2} + \beta_{G2E} \mu_E)I(G = 2) + \beta_Z Z}{\sqrt{(\beta_{G1E}^2 I(G = 1) \sigma_E^2 +\beta_{G2E}^2 I(G = 2) \sigma_E^2 + \beta_E^2 \sigma_E^2 + 1)}} \\
&= \gamma_0 + \gamma_1 I(G = 1) + \gamma_2 I(G = 2) + \gamma_Z Z + \gamma_{Z1G} I(G = 1) \times Z + \gamma_{Z2G} I(G = 2) \times Z
\end{aligned}
\end{equation}

where the new parameters \(\gamma_{Z1G}\) and \(\gamma_{Z2G}\) will be
defined as:
\[\gamma_{Z1G} = \frac{\beta_Z}{\sqrt{\beta_E^2\sigma_E^2+\beta_{G1E}^2\sigma_E^2 +1}}-\frac{\beta_Z}{\sqrt{\beta_E^2\sigma_E^2 +1}} \]
and:
\[\gamma_{Z2G} = \frac{\beta_Z}{\sqrt{\beta_E^2\sigma_E^2+\beta_{G2E}^2\sigma_E^2 +1}}-\frac{\beta_Z}{\sqrt{\beta_E^2\sigma_E^2 +1}} \]

In other words, a Genotypic Model with an missing interaction can still
be written as a linear function of these two indicator functions of G
because of the extra regression parameter. However, ignoring this
environment to gene interaction will create an artificial interaction
between gene and the covariate Z. Since the interaction effects are zero
if and only if the covariate Z has constant slopes across different
genotypes, we can test the environmental interaction by testing the null
hypothesis \(H_0: \gamma_{Z1G} = \gamma_{Z2G} = 0\), using either wald
test, likelihood ratio test or score test.

The key in this method is to test the equal slopes of the auxiliary
variable \(Z\). In order for this method to work, we need the following
assumption:

\begin{enumerate}
\item The auxiliary variable $Z_i$ is assumed to have no interaction effect with G in the model conditional on G, Z and E.
\item The auxiliary variable $Z_i$ is also assumed to have no interaction effect with E in the model conditional on G, Z and E.



\end{enumerate}

\end{document}
