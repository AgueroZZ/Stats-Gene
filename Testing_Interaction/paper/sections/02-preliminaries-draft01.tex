%!TEX root = ../GEtest-draft.tex


\section{Preliminaries}\label{sec:prelim}

\subsection{Latent formulation of Generalized linear Model}\label{subsec:latent}

Let $Y$ be the binary trait of interest, $G$ be the count of minor allele for the SNP of interest (so $G$ can take 0, 1, or 2) and $E$ be the environmental variable. The conventional method to model such binary response will be through the following generalized linear model (GLM):
\begin{equation}\label{equ:glm}
\begin{aligned}
P(Y = 1) = g^{-1}(\beta_0+\beta_GG + \beta_EE + \beta_{GE}GE)
\end{aligned}
\end{equation}
where the function $g^{-1}$ refers to the inverse of link function of the GLM. If a logistic regression is used, $g^{-1}$ will be the logistic function. If a probit regression being used, $g^{-1}$ will be the cumulative distribution function (CDF) of the standard normal distribution.


An equivalent formulation of the model above will be through the latent variable formulation. Define $Y^*$ be a latent variable that cannot be directly observed, generated by the following model:
\begin{equation}\label{equ:latentgenerating1}
\begin{aligned}
Y^* = \beta_0+\beta_GG + \beta_EE + \beta_{GE}GE + \epsilon
\end{aligned}
\end{equation}
the random error $\epsilon$ is assumed to be independent of all the covariates in the model, and can either have logistic distribution or have normal distribution, depending on whether the model \ref{equ:glm} is logistic or probit.


The latent variable $Y^*$ is not observable, but it generates the binary observations $Y$ in the following way:
\begin{equation}\label{equ:latentgenerating2}
\begin{aligned}
Y = \mathbb{I}\{Y^* > 0\}
\end{aligned}
\end{equation}
In other words, the response variable $Y$ can be viewed as an indicator variable defined based on the magnitude of the latent variable.


Assuming that the probit model \ref{equ:latentgenerating1} with random error $\epsilon \sim N(0,\sigma_\epsilon^2)$ generates the binary response variable, then the conditional probability of observing a case (i.e. Y = 1) can be computed as:
\begin{equation}\label{equ:computationOfProb}
\begin{aligned}
\text{P}(Y=1|G,E) &= \Phi\bigg(\frac{\E(Y^*|G,E)}{\sqrt{\text{Var}(Y^*|G,E)}}\bigg)\\
&= \Phi\bigg(\frac{\beta_0+\beta_GG+\beta_EE + \beta_{GE}GE}{\sigma_\epsilon}\bigg)\\
&= \Phi\big(\tilde{\beta_0}+\tilde{\beta_G}G+\tilde{\beta_E}E + \tilde{\beta}_{GE} GE\big)
\end{aligned}
\end{equation}
Where the parameter $\tilde{\beta}$ denotes $\frac{\beta}{\sigma_\epsilon}$. The parameter $\sigma_\epsilon$ will not be identifiable in the model, since any scalar multiplication simultaneously on $\beta$ and $\sigma_\epsilon$ will yield the same value of $\tilde{\beta}$. Therefore, in probit regression, the regression parameters actually refer to $\tilde{\beta}$ instead of $\beta$. In the rest of this work, unless stated otherwise, we will assume the random error has been properly standardized so $\tilde{\beta} = \beta$ for all the regression parameters.


A significant advantage of the probit model over the logistic model is that if $E \sim N(0,\sigma_E^2)$ is independent of both the random error $\epsilon$ and the SNP of interest $G$, the resulting model conditional on $G$ will still be a valid probit model. This result holds because $E + \epsilon$ will still be a normally distributed random error independent of $G$. However, if $\epsilon$ follows a logistic distribution instead, the same result will not necessarily hold even if $E$ also follows a logistic distribution. Therefore, the resulting model with a missing GxE interaction is not easily tractable if a logistic regression is used instead.



\subsection{Indirect testing of interaction effect using Levene's method}

The goal of indirect testing of interaction effect is to assess whether $\beta_{GE} = 0$ in model \ref{equ:latentgenerating1}, without using information of the environmental variable $E$. Assume for now that the latent variable $Y^*$ can be observed, and the unknown environmental variable $E$ is generated from $N(0,\sigma_E^2)$, independent of both the SNP $G$ and the random error $\epsilon$. Furthermore, assume that the true underlying model that generates the binary response variable is a probit model, and the random error $\epsilon$ follows distribution $N(0,\sigma_\epsilon^2)$.


With this specification, the problem reduces to the indirect testing of interaction effect when the "response" variable is quantative (i.e. $Y^*$ is quantative), instead of binary. As \citet{gjlst} have proposed, the indirect testing problem in this case can be done using the notion of generalized Levene test. If the interaction effect $\beta_{GE} \neq 0$, then the previous model \ref{equ:latentgenerating1} can be reduced to the following heteroskedastic linear regression model:

\begin{equation}\label{equ:heteroskedasticLM}
\begin{aligned}
Y^* = \beta_0 + \beta_GG + \epsilon_G
\end{aligned}
\end{equation}
where the new random error $\epsilon_G$ will have its variance dependent on $G$, in the following way:

\begin{equation}\label{equ:heteroskedasticForm}
\begin{aligned}
\text{Var}(\epsilon_G | G = g) = \text{Var}(Y^* | G = g) = (\beta_E \sigma_E + \beta_{GE} g)^2 + \sigma_\epsilon^2
\end{aligned}
\end{equation}
Equation \ref{equ:heteroskedasticForm} implies that the variance of $Y^*$ will be constant across different genotypic groups if and only if $\beta_{GE} = 0$. Therefore, the Levene-type test proposed in \citep{gjlst,jlst} can indirectly test for interaction by testing on the hypothesis of constant variance across genotypic groups.

However, the Levene-type method described above will not work for this problem, because the latent variable $Y^*$ is not \textit{observable} in practice. The observable traits $Y$ is linked to $\V(Y^*|G)$ only through $\E(Y^*|G)/\sqrt{\V(Y^*|G)}$, and hence the quantity $\V(Y^*|G)$ will not be identifiable from data of $Y$.